Il grassetto negli attributi indica che quell'attributo è o fa parte di una chiave. \\
Il simbolo $\gets$ identifica una generalizzazione completa, cioè l'\textbf{entita'} a sinistra è una generalizzazione completa delle identità che stanno a destra.

\begin{verse}
	L'\textbf{entita'} \emph{ProduzioneVino} è una generalizzazione completa delle entita' \emph{IngrMateriePrime, Pigiatura, Fermentazione, Vinificazione, Svinatura} che sono tutte prive di attributi.
\end{verse}
\begin{verse}
	Le \textbf{entita'} \emph{ProduzioneVino, Imbottigliamento, NegozioInterno, MagBianco, MagRosso, MagRosato, MagSpumante} sono prive di attributi.
\end{verse}

\vspace{0.5cm}
\begin{center}
	\begin{tabular}{P{4cm}P{2cm}P{8cm}}
		\multicolumn{3}{c}{\textbf {\large {Vigneto}}} \\
		\toprule
		\rowcolor[rgb]{.929, .929, .929} Attributo & Tipo & Descrizione \\
		\midrule
		\textbf{ID} & VARCHAR &  Identifica univocamente il vigneto\\
		\midrule
		Terreno & VARCHAR & Tipo di terreno del vigneto \\
		\midrule
		Indirzzo & VARCHAR &  Identifica la posizione geografica del vigneto.  Attributo composto: stato, città, provincia, cap, via, numero civico\\
		\bottomrule
	\end{tabular}
	
	\vspace{0.5cm}
	
	\begin{tabular}{P{4cm}P{2cm}P{8cm}}
		\multicolumn{3}{c}{\textbf {\large {Vigna}}} \\
		\toprule
		\rowcolor[rgb]{.929, .929, .929} Attributo & Tipo & Descrizione \\
		\midrule
		\textbf{ID} & VARCHAR &  Identifica univocamente la vigna\\
		\midrule
		DataPiantagione & DATE & Data piantagione della vigna \\
		\bottomrule
	\end{tabular}

		\vspace{0.5cm}
	
	\begin{tabular}{P{4cm}P{2cm}P{8cm}}
		\multicolumn{3}{c}{\textbf {\large {Uva}}} \\
		\toprule
		\rowcolor[rgb]{.929, .929, .929} Attributo & Tipo & Descrizione \\
		\midrule
		\textbf{Tipologia} & VARCHAR &  Identifica univocamente il tipo d'uva\\
		\midrule
		\textbf{AnnoRaccolta} & INTEGER & Anno in cui viene raccolta l'uva \\
		\midrule
		Colore & VARCHAR & Identifica il colore dell'uva \\
		\bottomrule
	\end{tabular}

		\vspace{0.5cm}
	
	\begin{tabular}{P{4cm}P{2cm}P{8cm}}
		\multicolumn{3}{c}{\textbf {\large {FornitoreUva}}} \\
		\toprule
		\rowcolor[rgb]{.929, .929, .929} Attributo & Tipo & Descrizione \\
		\midrule
		PrezzoAlKg & DECIMAL & Identifica il prezzo di vendita dell'uva in kg da parte del fornitore \\
		\bottomrule
	\end{tabular}
	
	\vspace{0.5cm}

	\begin{tabular}{P{4cm}P{2cm}P{8cm}}
	\multicolumn{3}{c}{\textbf {\large {FornitoreTappi}}} \\
	\toprule
	\rowcolor[rgb]{.929, .929, .929} Attributo & Tipo & Descrizione \\
	\midrule
	PrezzoAlTappo & DECIMAL & Identifica il prezzo di vendita di un tipo di tappo da parte del fornitore\\
	\bottomrule
\end{tabular}

\vspace{0.5cm}

\begin{tabular}{P{4cm}P{2cm}P{8cm}}
	\multicolumn{3}{c}{\textbf {\large {FornitoreBottiglie}}} \\
	\toprule
	\rowcolor[rgb]{.929, .929, .929} Attributo & Tipo & Descrizione \\
	\midrule
	PrezzoABottiglia & DECIMAL & Identifica il prezzo di vendita di una particolare bottiglia da parte del fornitore \\
	\bottomrule
\end{tabular}

	\vspace{0.5cm}
	
\begin{tabular}{P{4cm}P{2cm}P{8cm}}
	\multicolumn{3}{c}{\textbf {\large {Fornitore} $\gets$ (\emph{FornitoreUva, FornitoreTappi, FornitoreBottiglie})}} \\
	\toprule
	\rowcolor[rgb]{.929, .929, .929} Attributo & Tipo & Descrizione \\
	\midrule
	\textbf{Id} & VARCHAR &  Identifica univocamente il fornitore\\
	\midrule
	Nome & VARCHAR &  Identifica il nome commerciale del fornitore\\
	\bottomrule
\end{tabular}
	
	\vspace{0.5cm}
	
\begin{tabular}{P{4cm}P{2cm}P{8cm}}
	\multicolumn{3}{c}{\textbf {\large {Tappo}}} \\
	\toprule
	\rowcolor[rgb]{.929, .929, .929} Attributo & Tipo & Descrizione \\
	\midrule
	\textbf{Forma} & VARCHAR &  Rappresenta il tipo di forma del tappo\\
	\midrule
	\textbf{Materiale} & VARCHAR &  Rappresenta il tipo di materiale del tappo\\
	\midrule
	Quantità & INTEGER &  Rappresenta il numero di tappi per forma e colore posseduti dalla cantina\\
	\bottomrule
\end{tabular}

	\vspace{0.5cm}

\begin{tabular}{P{4cm}P{2cm}P{8cm}}
	\multicolumn{3}{c}{\textbf {\large {Bottiglia}}} \\
	\toprule
	\rowcolor[rgb]{.929, .929, .929} Attributo & Tipo & Descrizione \\
	\midrule
	\textbf{Colore} & VARCHAR &  Rappresenta il colore della bottiglia\\
	\midrule
	\textbf{Capacità} & VARCHAR &  Rappresenta la capacità della bottiglia\\
	\midrule
	Quantità & INTEGER &  Rappresenta il numero di bottiglie per capacità e colore possedute dalla cantina\\
	\bottomrule
\end{tabular}
	\vspace{0.5cm}

\begin{tabular}{P{4cm}P{2cm}P{8cm}}
	\multicolumn{3}{c}{\textbf {\large {Vino}}} \\
	\toprule
	\rowcolor[rgb]{.929, .929, .929} Attributo & Tipo & Descrizione \\
	\midrule
	\textbf{Nome} & VARCHAR & Identifica il nome del vino\\
	\midrule
	GradazioneAlcolica & TINYINT & Identifica il grado di alcol del vino\\
	\midrule
	TempoFermentazione & TINYINT & Rappresenta i giorni che sono serviti per la fermentazione del vino\\
	\midrule
	StatoProduzione & BOOLEAN & Rappresenta se il vino è ancora in produzione\\
	\bottomrule
\end{tabular}


	\vspace{0.5cm}

\begin{tabular}{P{4cm}P{2cm}P{8cm}}
	\multicolumn{3}{c}{\textbf {\large {BottigliaDiVino}}} \\
	\toprule
	\rowcolor[rgb]{.929, .929, .929} Attributo & Tipo & Descrizione \\
	\midrule
	\textbf{Nome} & VARCHAR &  Rappresenta il nome del vino a cui la bottiglia fa riferimento\\
	\midrule
	\textbf{Annata} & INTEGER &  Rappresenta l'anno della vendemmia dell'uva con cui è stato prodotto il vino\\
	\midrule
	Prezzo & DECIMAL &  Identifica il prezzo della bottiglia di vino\\
	\midrule
	Classificazione & VARCHAR & Tutela i consumatori su alcune caratteristiche del vino\\
	\midrule
	NumBottiglieVendute & INTEGER & Numero di bottiglie per nome e annata di vino vendute\\
	\midrule
	NumBottiglieProdotte & INTEGER &  Numero di bottiglie per nome e annata di vino prodotte\\
	\bottomrule
\end{tabular}


	\vspace{0.5cm}

\begin{tabular}{P{4cm}P{2cm}P{8cm}}
	\multicolumn{3}{c}{\textbf {\large {Ordine}}} \\
	\toprule
	\rowcolor[rgb]{.929, .929, .929} Attributo & Tipo & Descrizione \\
	\midrule
	\textbf{ID} & VARCHAR &  Identifica univocamente un ordine di vendita ricevuto\\
	\midrule
	PrezzoTotale & INTEGER &  Rappresenta il prezzo complessivo dell'ordine, formato da prezzo di spedizione più il costo delle bottigle acquistate\\
	\midrule
	Data & DATE &  Data in cui è stato effettuato l'ordine\\
	\bottomrule
\end{tabular}

	\vspace{0.5cm}

\begin{tabular}{P{4cm}P{2cm}P{8cm}}
	\multicolumn{3}{c}{\textbf {\large {Corriere}}} \\
	\toprule
	\rowcolor[rgb]{.929, .929, .929} Attributo & Tipo & Descrizione \\
	\midrule
	\textbf{ID} & VARCHAR &  Identifica univocamente il corriere\\
	\midrule
	Nome & VARCHAR &  Rappresenta il nome commerciale del corriere\\
	\bottomrule
\end{tabular}

	\vspace{0.5cm}


\begin{tabular}{P{4cm}P{2cm}P{8cm}}
	\multicolumn{3}{c}{\textbf {\large {Azienda} $\gets$ (\emph{NegozioInterno})}} \\
	\toprule
	\rowcolor[rgb]{.929, .929, .929} Attributo & Tipo & Descrizione \\
	\midrule
	NomeReferente & VARCHAR &  Rappresenta il nome del referente dell'azienda\\
	\midrule
	CognomeReferente & VARCHAR &  Rappresenta il cognome del referente dell'azienda\\
	\bottomrule
\end{tabular}

\vspace{0.5cm}

\begin{tabular}{P{4cm}P{2cm}P{8cm}}
	\multicolumn{3}{c}{\textbf {\large {Privato}}} \\
	\toprule
	\rowcolor[rgb]{.929, .929, .929} Attributo & Tipo & Descrizione \\
	\midrule
	Cognome & VARCHAR &  Rappresenta il cognome dell'acquirente privato\\
	\bottomrule
\end{tabular}

\vspace{0.5cm}

\begin{tabular}{P{4cm}P{2cm}P{8cm}}
	\multicolumn{3}{c}{\textbf {\large {Acquirente} $\gets$ (\emph{Azienda, Privato})}} \\
	\toprule
	\rowcolor[rgb]{.929, .929, .929} Attributo & Tipo & Descrizione \\
	\midrule
	\textbf{ID} & VARCHAR &  Identifica univocamente l'acquirente\\
	\midrule
	Nome & VARCHAR &  Rappresenta il nome dell'acquirente\\
	\midrule
	Telefono & VARCHAR &  Rappresenta il numero telefonico dell'acquirente\\
	\midrule
	Indirzzo & VARCHAR &  Identifica la posizione geografica cui risiede l'acquirente.  Attributo composto: stato, città, provincia, cap, via, numero civico\\
	\bottomrule
\end{tabular}

\vspace{0.5cm}

\begin{tabular}{P{4cm}P{2cm}P{8cm}}
	\multicolumn{3}{c}{\textbf {\large {Eventi}}} \\
	\toprule
	\rowcolor[rgb]{.929, .929, .929} Attributo & Tipo & Descrizione \\
	\midrule
	\textbf{Titolo} & VARCHAR &  Rappresenta il titolo dell'evento\\
	\midrule
	\textbf{NumEdizione} & INTEGER &  Rappresenta l'edizione dell'evento\\
	\bottomrule
\end{tabular}

\vspace{0.5cm}

\begin{tabular}{P{4cm}P{2cm}P{8cm}}
	\multicolumn{3}{c}{\textbf {\large {Partecipante}}} \\
	\toprule
	\rowcolor[rgb]{.929, .929, .929} Attributo & Tipo & Descrizione \\
	\midrule
	\textbf{ID} & VARCHAR &  Rappresenta univocamente il partecipante di un evento\\
	\midrule
	Nome & VARCHAR &  Rappresenta il nome del partecipante\\
	\midrule
	Cognome & VARCHAR &  Rappresenta il cognome del partecipante\\
	\midrule
	Età & TINYINT &  Rappresenta l'età del partecipante, che devo essere maggiore o uguale a 18 anni\\
	\bottomrule
\end{tabular}

\vspace{0.5cm}


\begin{tabular}{P{4cm}P{2cm}P{8cm}}
	\multicolumn{3}{c}{\textbf {\large {LineaProduttiva} $\gets$ (\emph{ProduzioneVino, Imbottigliamento})}} \\
	\toprule
	\rowcolor[rgb]{.929, .929, .929} Attributo & Tipo & Descrizione \\
	\midrule
	\textbf{ID} & VARCHAR &  Rappresenta univocamente la linea produttiva\\
	\bottomrule
\end{tabular}

\vspace{0.5cm}

\begin{tabular}{P{4cm}P{2cm}P{8cm}}
	\multicolumn{3}{c}{\textbf {\large {Magazzino} $\gets$ (\emph{MagBianco, MagRosso, MagRosato, MagSpumante})}} \\
	\toprule
	\rowcolor[rgb]{.929, .929, .929} Attributo & Tipo & Descrizione \\
	\midrule
	\textbf{NumBottiglie} & INTEGER &  Rappresenta la quantità delle bottiglie di vino, divise per tipologia di colore, contenute nel magazzino\\
	\bottomrule
\end{tabular}

\vspace{0.5cm}

\begin{tabular}{P{4cm}P{2cm}P{8cm}}
	\multicolumn{3}{c}{\textbf {\large {Dipendente}}} \\
	\toprule
	\rowcolor[rgb]{.929, .929, .929} Attributo & Tipo & Descrizione \\
	\midrule
	\textbf{CodFiscale} & VARCHAR &  Rappresenta univocamente il dipendente\\
	\midrule
	Nome & VARCHAR & Rappresenta il nome del dipendente \\
	\midrule
	Cognome & VARCHAR & Rappresenta il congnome del dipendente \\
	\midrule
	Stipendio & INTEGER & Rappresenta lo stipendio mensile del dipendente \\
	\bottomrule
\end{tabular}

\vspace{0.5cm}

\begin{tabular}{P{4cm}P{2cm}P{8cm}}
	\multicolumn{3}{c}{\textbf {\large {Macchinario}}} \\
	\toprule
	\rowcolor[rgb]{.929, .929, .929} Attributo & Tipo & Descrizione \\
	\midrule
	\textbf{ID} & VARCHAR &  Rappresenta univocamente il macchinario\\
	\midrule
	Nome & VARCHAR & Rappresenta il nome commerciale del macchinario \\
	\midrule
	DataProManute & DATE & Rappresenta la data prossima della manutenzione \\
	\midrule
	DataUltManute & DATE & Rappresenta la data dell'ultima manutenzione \\
	\bottomrule
\end{tabular}

\vspace{0.5cm}

\begin{tabular}{P{4cm}P{2cm}P{8cm}}
	\multicolumn{3}{c}{\textbf {\large {AziendaManutenzione}}} \\
	\toprule
	\rowcolor[rgb]{.929, .929, .929} Attributo & Tipo & Descrizione \\
	\midrule
	\textbf{ID} & VARCHAR &  Rappresenta univocamente l'azienda di manutenzione dei macchinari\\
	\midrule
	Nome & VARCHAR & Rappresenta il nome commerciale dell'azienda che fa manutenzione ai macchinari\\
	\bottomrule
\end{tabular}

\end{center}