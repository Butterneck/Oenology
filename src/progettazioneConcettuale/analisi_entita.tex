Gli attributi contraddistinti dal carattere tipografico \textbf{\emph{grassetto}} sono (o fanno parte) della chiave primaria della relativa tabella. \\
Il simbolo $\gets$ identifica una generalizzazione completa, cioè la \textbf{prima entità} è una generalizzazione \emph{completa} delle entità seguenti.\\
Il simbolo $\Leftarrow$ identifica una generalizzazione parziale, cioè la \textbf{prima entità} è una generalizzazione \emph{parziale} delle entità seguenti.

\begin{verse}
	Le entità \emph{Fornitore, ProduzioneVino, Imbottigliamento, NegozioInterno, MagBianco, MagRosso, MagRosato, MagSpumante, AzManutenzione, IngrMateriePrime, Pigiatura, Fermentazione, Vinificazione, Svinatura, FornitoreTappi, FornitureUva, FornitoreBottiglie} vengono omesse poichè prive di attributi.
\end{verse}
\begin{verse}
	\textbf{ProduzioneVino} $\gets$ \emph{IngrMateriePrime, Pigiatura, Fermentazione, Vinificazione, Svinatura}.
\end{verse}
\begin{verse}
	\textbf{Fornitore} $\gets$ \emph{FornitoreTappi, FornitureUva, FornitoreBottiglie}.
\end{verse}

\vspace{0.3cm}
\begin{center}
	\begin{tabular}{P{4cm}P{2cm}P{8cm}}
		\multicolumn{3}{c}{\textbf {\large {Vigneto}}}                                                                                                                 \\
		\toprule
		\rowcolor[rgb]{.929, .929, .929} Attributo & Tipo    & Descrizione                                                                                             \\
		\midrule
		\textbf{Id}                                & INTEGER & Identifica univocamente il vigneto                                                                      \\
		\midrule
		Indirzzo                                   & VARCHAR & Posizione geografica del vigneto.  Attributo composto: stato, città, provincia, cap, via, numero civico \\
		\bottomrule
	\end{tabular}

	\vspace{0.3cm}

	\begin{tabular}{P{4cm}P{2cm}P{8cm}}
		\multicolumn{3}{c}{\textbf {\large {TipoUva}}}                         \\
		\toprule
		\rowcolor[rgb]{.929, .929, .929} Attributo & Tipo    & Descrizione     \\
		\midrule
		\textbf{Nome}                              & VARCHAR & Nome dell'uva   \\
		\midrule
		Colore                                     & VARCHAR & Colore dell'uva \\
		\bottomrule
	\end{tabular}

	\vspace{0.3cm}
	\begin{tabular}{P{4cm}P{2cm}P{8cm}}
		\multicolumn{3}{c}{\textbf {\large {Uva}}}                                              \\
		\toprule
		\rowcolor[rgb]{.929, .929, .929} Attributo & Tipo    & Descrizione                      \\
		\midrule
		Annata                                     & INTEGER & Anno in cui viene raccolta l'uva \\
		\bottomrule
	\end{tabular}

	\vspace{0.3cm}


	\begin{tabular}{P{4cm}P{2cm}P{8cm}}
		\multicolumn{3}{c}{\textbf {\large {Tappo}}}                                                                        \\
		\toprule
		\rowcolor[rgb]{.929, .929, .929} Attributo & Tipo    & Descrizione                                                  \\
		\midrule
		Forma                                      & VARCHAR & Tipo di forma del tappo                                      \\
		\midrule
		Materiale                                  & VARCHAR & Tipo di materiale del tappo                                  \\
		\midrule
		Quantità                                   & INTEGER & Numero di tappi di un determinato tipo non ancora utilizzati \\
		\bottomrule
	\end{tabular}

	\vspace{0.3cm}

	\begin{tabular}{P{4cm}P{2cm}P{8cm}}
		\multicolumn{3}{c}{\textbf {\large {Bottiglia}}}                                                                        \\
		\toprule
		\rowcolor[rgb]{.929, .929, .929} Attributo & Tipo    & Descrizione                                                      \\
		\midrule
		Colore                                     & VARCHAR & Colore della bottiglia                                           \\
		\midrule
		Capacità                                   & VARCHAR & Capacità della bottiglia                                         \\
		\midrule
		Quantità                                   & INTEGER & Numero di bottiglie di un determinato tipo non ancora utilizzate \\
		\bottomrule
	\end{tabular}
	\vspace{0.3cm}

	\begin{tabular}{P{4cm}P{2cm}P{8cm}}
		\multicolumn{3}{c}{\textbf {\large {MateriaPrima} $\gets$ (\emph{Uva, Tappo, Bottiglia})}}      \\
		\toprule
		\rowcolor[rgb]{.929, .929, .929} Attributo & Tipo    & Descrizione                              \\
		\midrule
		\textbf{Id}                                & INTEGER & Identifica univocamente la materia prima \\
		\bottomrule
	\end{tabular}

	\vspace{0.3cm}

	\begin{tabular}{P{4cm}P{2cm}P{8cm}}
		\multicolumn{3}{c}{\textbf {\large {Vino}}}                                                                   \\
		\toprule
		\rowcolor[rgb]{.929, .929, .929} Attributo & Tipo    & Descrizione                                            \\
		\midrule
		\textbf{Nome}                              & VARCHAR & Identifica univocamente un tipo di vino                \\
		\midrule
		GradazioneAlcolica                         & TINYINT & Gradazione alcolica del vino                           \\
		\midrule
		TempoFermentazione                         & TINYINT & Numero di giorni di fermentazione del vino             \\
		\midrule
		StatoProduzione                            & BOOLEAN & Identifica se il vino è ancora in produzione           \\
		\midrule
		Classificazione                            & ENUM    & Caratteristiche del vino per la tutela dei consumatori \\
		\bottomrule
	\end{tabular}

	\vspace{0.3cm}

	\begin{tabular}{P{4cm}P{2cm}P{8cm}}
		\multicolumn{3}{c}{\textbf {\large {Corriere}}}                                            \\
		\toprule
		\rowcolor[rgb]{.929, .929, .929} Attributo & Tipo    & Descrizione                         \\
		\midrule
		\textbf{Id}                                & INTEGER & Identifica univocamente il corriere \\
		\midrule
		Nome                                       & VARCHAR & Nome del corriere                   \\
		\bottomrule
	\end{tabular}


	\vspace{0.3cm}

	\begin{tabular}{P{4cm}P{2cm}P{8cm}}
		\multicolumn{3}{c}{\textbf {\large {BottigliaDiVino}}}                                                       \\
		\toprule
		\rowcolor[rgb]{.929, .929, .929} Attributo & Tipo    & Descrizione                                           \\
		\midrule
		\textbf{Nome}                              & VARCHAR & Identifica il nome del vino contenuto nella bottiglia \\
		\midrule
		\textbf{Annata}                            & INTEGER & Identifica l'anno di imbottigliamento della bottiglia \\
		\midrule
		Prezzo                                     & DECIMAL & Prezzo della bottiglia di vino                        \\
		\midrule
		NumBottiglieVendute                        & INTEGER & Numero di bottiglie di un determinato tipo vendute    \\
		\midrule
		NumBottiglieProdotte                       & INTEGER & Numero di bottiglie di un determinato tipo prodotte   \\
		\bottomrule
	\end{tabular}


	\vspace{0.3cm}

	\begin{tabular}{P{4cm}P{2cm}P{8cm}}
		\multicolumn{3}{c}{\textbf {\large {Ordine}}}                                                                                                                           \\
		\toprule
		\rowcolor[rgb]{.929, .929, .929} Attributo & Tipo    & Descrizione                                                                                                      \\
		\midrule
		\textbf{Id}                                & INTEGER & Identifica univocamente un ordine di vendita ricevuto                                                            \\
		\midrule
		PrezzoTotale                               & INTEGER & Prezzo complessivo dell'ordine, uguale alla somma del prezzo di spedizione e del costo delle bottigle acquistate \\
		\midrule
		Data                                       & DATE    & Data in cui è stato effettuato l'ordine                                                                          \\
		\bottomrule
	\end{tabular}

	\vspace{0.3cm}


	\begin{tabular}{P{4cm}P{2cm}P{8cm}}
		\multicolumn{3}{c}{\textbf {\large {Azienda} $\gets$ (\emph{NegozioInterno, Fornitore, AzManutenzione})}}                                                                  \\
		\toprule
		\rowcolor[rgb]{.929, .929, .929} Attributo & Tipo    & Descrizione                                                                                                         
		\\	\midrule
		\textbf{PartitaIVA}                                 & VARCHAR & Identifica la partita IVA dell'azienda                                                                              \\
		\midrule
		NomeReferente                              & VARCHAR & Nome del referente dell'azienda                                                                                     \\
		\midrule
		CognomeReferente                           & VARCHAR & Cognome del referente dell'azienda                                                                                   \\
		\midrule
		Nome                                       & VARCHAR & Nome dell'azienda                                                                                                   \\
		\midrule
		Telefono                                   & VARCHAR & Numero telefonico dell'azienda                                                                                      \\
		\midrule
		Indirzzo                                   & VARCHAR & Posizione geografica della sede dell'azienda.  Attributo composto: stato, città, provincia, cap, via, numero civico \\
		\midrule
		Email                                      & VARCHAR & Email dell'azienda                                                                                                  \\
		\bottomrule
	\end{tabular}

	\vspace{0.3cm}

	\begin{tabular}{P{4cm}P{2cm}P{8cm}}
		\multicolumn{3}{c}{\textbf {\large {Privato}}}                                                                                                                                                                       \\
		\toprule
		\rowcolor[rgb]{.929, .929, .929} Attributo & Tipo    & Descrizione                                                                                                                                                   \\
		\midrule
		Nome                                       & VARCHAR & Nome del privato                                                                                                                                              \\
		\midrule
		Telefono                                   & VARCHAR & Numero telefonico del privato                                                                                                                                 \\
		\midrule
		Indirzzo                                   & VARCHAR & Posizione geografica cui risiede il privato, corrisponde con l'indirizzo di spedizione.  Attributo composto: stato, città, provincia, cap, via, numero civico \\
		\midrule
		Email                                      & VARCHAR & Email del privato                                                                                                                                             \\
		\midrule
		Cognome                                    & VARCHAR & Cognome del privato                                                                                                                                           \\
		\bottomrule
	\end{tabular}

	\vspace{0.3cm}

	\begin{tabular}{P{4cm}P{2cm}P{8cm}}
		\multicolumn{3}{c}{\textbf {\large {Acquirente} $\gets$ (\emph{Privato}), \large{Acquirente} $\Leftarrow$ (\emph{Azienda})}} \\
		\toprule
		\rowcolor[rgb]{.929, .929, .929} Attributo & Tipo    & Descrizione                                                           \\
		\midrule
		\textbf{Id}                                & INTEGER & Identifica univocamente un acquirente                                 \\
		\bottomrule
	\end{tabular}

	\vspace{0.3cm}

	\begin{tabular}{P{4cm}P{2cm}P{8cm}}
		\multicolumn{3}{c}{\textbf {\large {Evento}}}                                         \\
		\toprule
		\rowcolor[rgb]{.929, .929, .929} Attributo & Tipo    & Descrizione                    \\
		\midrule
		\textbf{Titolo}                            & VARCHAR & Titolo dell'evento             \\
		\midrule
		\textbf{Edizione}                          & INTEGER & Numero di edizione dell'evento \\
		\bottomrule
	\end{tabular}

	\vspace{0.3cm}


	\begin{tabular}{P{4cm}P{2cm}P{8cm}}
		\multicolumn{3}{c}{\textbf {\large {LineaProduttiva} $\gets$ (\emph{ProduzioneVino, Imbottigliamento, Magazzino})}} \\
		\toprule
		\rowcolor[rgb]{.929, .929, .929} Attributo & Tipo    & Descrizione                                                  \\
		\midrule
		\textbf{Id}                                & INTEGER & Identifica univocamente una linea produttiva                 \\
		\bottomrule
	\end{tabular}

	\vspace{0.3cm}

	\begin{tabular}{P{4cm}P{2cm}P{8cm}}
		\multicolumn{3}{c}{\textbf {\large {Magazzino} $\gets$ (\emph{MagBianco, MagRosso, MagRosato, MagSpumante})}}                                                   \\
		\toprule
		\rowcolor[rgb]{.929, .929, .929} Attributo & Tipo    & Descrizione                                                                                              \\
		\midrule
		NumBottiglie & INTEGER & Rappresenta la quantità delle bottiglie di vino, divise per tipologia di colore, contenute nel magazzino \\
		\bottomrule
	\end{tabular}

	\vspace{0.3cm}

	\begin{tabular}{P{4cm}P{2cm}P{8cm}}
		\multicolumn{3}{c}{\textbf {\large {Dipendente}}}                                            \\
		\toprule
		\rowcolor[rgb]{.929, .929, .929} Attributo & Tipo    & Descrizione                           \\
		\midrule
		\textbf{CodiceFiscale}                     & VARCHAR & Identifica univocamente un dipendente \\
		\midrule
		Nome                                       & VARCHAR & Nome del dipendente                   \\
		\midrule
		Cognome                                    & VARCHAR & Cognome del dipendente                \\
		\bottomrule
	\end{tabular}

	\vspace{0.3cm}

	\begin{tabular}{P{4cm}P{2cm}P{8cm}}
		\multicolumn{3}{c}{\textbf {\large {Macchinario}}}                                            \\
		\toprule
		\rowcolor[rgb]{.929, .929, .929} Attributo & Tipo    & Descrizione                            \\
		\midrule
		\textbf{Id}                                & INTEGER & Identifica univocamente un macchinario \\
		\midrule
		Nome                                       & VARCHAR & Nome commerciale del macchinario       \\
		\midrule
		DataProssimaManutenzione                   & DATE    & Data prossima della manutenzione       \\
		\midrule
		DataAcquisto                               & DATE    & Data acquisto macchinario              \\
		\bottomrule
	\end{tabular}

\end{center}