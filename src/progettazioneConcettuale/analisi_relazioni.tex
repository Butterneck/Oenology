\begin{itemize}
	\item \underline{Vigneto - TipoUva}: \textbf{Proviene}
	
	\begin{itemize}
		\item Un vigneto produce solamente una tipologia d'uva $(1,1)$.
		\item Un tipo di uva può provenire da più vigneti $(1,N)$.
	\end{itemize}
	
\end{itemize}

\begin{itemize}
	\item \underline{Uva - Vino}: \textbf{Prodotto}
	
	\begin{itemize}
		\item Da un tipo di uva (di una determinata annata) possono essere stati prodotti più tipi di vino o nessun tipo di vino $(0,N)$*.
		\item Un vino è prodotto da un solo tipo d'uva (di una determinata annata) $(1,1)$.
	\end{itemize}
	
\end{itemize}

\begin{verse}
	*\emph{(0,N) perchè una nuova tipologia d'uva può essere appena stata acquistata e quindi non è stato prodotto ancora nessun vino.}
\end{verse}


\begin{itemize}
	\item \underline{MateriaPrima - Fornitore}: \textbf{Fornitura}*
	
	\begin{itemize}
		\item Una Materia prima viene fornita da un solo fornitore $(1,1)$.
		\item Un fornitore fornisce una o più tipologie di materie prime $(1,N)$.
	\end{itemize}
	
\end{itemize}

\begin{verse}
*\emph{Nella relazione sono presenti gli attributi \textbf{DataAcquisto, Prezzo, Quantità} perchè una fornitura può essere effettuata più volte, con data, prezzo e quantità differenti.}
\end{verse}


\begin{itemize}
	\item \underline{BottigliaDiVino - Vino}: \textbf{TipoVino}
	
	\begin{itemize}
		\item Una bottiglia di vino contiene solamente una tipologia di vino $(1,1)$.
		\item Un vino può essere contenuto in più bottiglie di vino o in nessuna $(0,N)$*.
	\end{itemize}
	
\end{itemize}

\begin{verse}
	*\emph{$(0,N)$ perchè un determinato tipo di vino può essere appena stato prodotto e quindi non essere ancora stato imbottigliato, oppure possono essere terminate le bottiglie che lo contenevano.}
\end{verse}

\begin{itemize}
	\item \underline{BottigliaDiVino - Tappo}: \textbf{TipoTappo}
	
	\begin{itemize}
		\item Una bottiglia di vino ha solamente un tappo $(1,1)$.
		\item Un tipo di tappo può appartenere a più bottiglie di vino o a nessuna $(0,N)$*.
	\end{itemize}
	
\end{itemize}

\begin{verse}
	*\emph{(0,N) perchè una tipologia di tappo può essere appena stata acquistata e quindi non essere ancora assegnata a nessuna bottiglia di vino.}
\end{verse}

\begin{itemize}
	\item \underline{BottigliaDiVino - Bottiglia}: \textbf{TipoBottiglia}
	
	\begin{itemize}
		\item Una bottiglia di vino è formata da una sola bottiglia $(1,1)$.
		\item Un tipo di bottiglia può appartenere a più bottiglie di vino o a nessuna $(0,N)$*.
	\end{itemize}
	
\end{itemize}

\begin{verse}
	*\emph{(0,N) perchè una tipologia di bottiglia può essere appena stata acquistata e quindi non essere ancora assegnata a nessuna bottiglia di vino.}
\end{verse}

\begin{itemize}
	\item \underline{BottigliaDiVino - Magazzino}: \textbf{Conservata}
	
	\begin{itemize}
		\item Una tipologia di bottiglia di vino è conservata in un unico magazzino $(1,1)$
		\item Un magazzino contiene da una a molteplici tipologie di vino, poichè la loro divisione avviene per colore e non per annata $(1,N)$.
	\end{itemize}
	
\end{itemize}

\begin{itemize}
	\item \underline{BottigliaDiVino - Ordine}: \textbf{Dettaglio}*
	
	\begin{itemize}
		\item Un tipo di bottiglia di vino può appartenere a zero o più ordini di vendita $(0,N)$.
		\item Un ordine di vendita contiene uno o più tipi di bottiglie di vino acquistate $(1,N)$.
	\end{itemize}
	
\end{itemize}

\begin{verse}
	*\emph{Nella relazione è presente l'attributo \textbf{QuantitàBottiglie} poichè è necessario tenere traccia del numero di bottiglie di vino per tipologia richieste dall'ordine.}
\end{verse}

\begin{itemize}
	\item \underline{Ordine - Corriere}: \textbf{Spedizione}*
	
	\begin{itemize}
		\item Un ordine viene spedito da un singolo corriere $(1,1)$.
		\item Un corriere può spedire uno o più ordini $(1,N)$.
	\end{itemize}
	
\end{itemize}

\begin{verse}
	*\emph{Nella relazione sono presenti gli attributi \textbf{DataSpedizione, Prezzo, DataArrivo} perchè dell'ordine è importante tracciare la data in cui è stato spedito l'ordine, il prezzo della spedizione e la data di consegna dell'ordine.}
\end{verse}

\begin{itemize}
	\item \underline{Ordine - Acquirente}: \textbf{Venduta}
	
	\begin{itemize}
		\item Un ordine è effettuato da un singolo acquirente $(1,1)$.
		\item Un acquirente può effettuare uno o più ordini $(1,N)$.
	\end{itemize}
	
\end{itemize}

\begin{itemize}
	\item \underline{NegozioInterno - Evento}: \textbf{Ospita}*
	
	\begin{itemize}
		\item Un negozio interno può ospitare o nessuno o molteplici eventi $(0,N)$.
		\item Un evento può essere ospitato in uno o in molteplici negozi $(1,N)$.
	\end{itemize}
	
\end{itemize}

\begin{verse}
	*\emph{Nella relazione è presente l'attributo \textbf{Data} perchè un evento può essere ospitato in un negozio solamente in una singola data.}
\end{verse}


\begin{itemize}
	\item \underline{Evento - Vino}: \textbf{TemaVino}
	
	\begin{itemize}
		\item Un evento può avere uno o molteplici vini come tema $(1,N)$.
		\item Un vino può essere essere il tema di nessuno o molteplici eventi $(0,N)$.
	\end{itemize}
	
\end{itemize}

\begin{itemize}
	\item \underline{LineaProduttiva - Dipendente}: \textbf{Turno}*
	
	\begin{itemize}
		\item In una linea produttiva possono lavorare uno o molteplici dipendenti $(1,N)$.
		\item Un dipendete può lavorare in una sola linea produttiva $(1,1)$.
	\end{itemize}
	
\end{itemize}

\begin{verse}
	*\emph{Nella relazione sono presenti gli attributi \textbf{InizioTurno, FineTurno} perchè è utile rappresentare il giorno e l'orario in cui un dipendente lavora. Entrambi gli attributi sono rappresentati attraverso il tipo DATETIME.}
\end{verse}

\begin{itemize}
	\item \underline{LineaProduttiva - Dipendente}: \textbf{Diretto}
	
	\begin{itemize}
		\item Una linea produttiva è diretta da un solo dipendente $(1,1)$.
		\item Un dipendete può dirigere più linee produttive o nessuna $(0,N)$.
	\end{itemize}
	
\end{itemize}

\begin{itemize}
	\item \underline{Dipendente - Dipendente}: \textbf{Referisce}
	
	\begin{itemize}
		\item Un dipendente può essere riferito da uno o nessun altro dipendente $(0,1)$.
		\item Un dipendente può riferire molteplici dipendenti o nessuno $(0,N)$.
	\end{itemize}
	
\end{itemize}

\begin{itemize}
	\item \underline{LineaProduttiva - Macchinario}: \textbf{Utilizzo}
	
	\begin{itemize}
		\item In una linea produttiva possono essere utilizzati molteplici macchinari oppure nessuno $(0,N)$.
		\item Un macchinario può appartenere solamente ad una linea produttiva $(1,1)$.
	\end{itemize}
	
\end{itemize}

\begin{itemize}
	\item \underline{Macchinario - AziendaManutenzione}: \textbf{Manutenzione}*
	
	\begin{itemize}
		\item Un macchinario può aver ricevuto molteplici manutenzioni o nessuna $(0,N)$.
		\item Un'azienda di manutenzione può revisionare uno o molteplici macchinari $(1,N)$.
	\end{itemize}
	
\end{itemize}

\begin{verse}
	*\emph{Nella relazione sono presenti gli attributi \textbf{Costo} e \textbf{Data} che identificano il costo della singola manutenzione e in quale data è stata effettuata.}
\end{verse}

