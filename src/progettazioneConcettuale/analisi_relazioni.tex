\begin{itemize}
	\item \underline{Vigneto - Vigna}: \textbf{Coltivata}
	
	\begin{itemize}
		\item Nel vigneto sono presenti uno o più vigne (1,N)
		\item Una vigna è presente in un solo vigneto (1,1)
	\end{itemize}
	
\end{itemize}

\begin{itemize}
	\item \underline{Vigna - Uva}: \textbf{Proviene}
	
	\begin{itemize}
		\item La vigna contiene solo una tipologia d'uva (1,1)
		\item Un tipo di uva può provenire da più vigne (1,N)
	\end{itemize}
	
\end{itemize}

\begin{itemize}
	\item \underline{Uva - TipoUva}: \textbf{TipologiaUva}
	
	\begin{itemize}
		\item Un tipo d'annata d'uva proviene da un solo tipo d'uva (1,1)
		\item Un tipo d'uva può rappresentare uva d'annata diversa (1,N)
	\end{itemize}
	
\end{itemize}

\begin{itemize}
	\item \underline{Uva - Vino}: \textbf{Prodotto}
	
	\begin{itemize}
		\item Da un tipo d'annata d'uva possono essere prodotti o nessuno o più tipi di vino (0,N)*
		\item Un vino è prodotto da un solo tipo d'annata d'uva (1,1)
	\end{itemize}
	
\end{itemize}

\begin{verse}
	*\emph{(0,N) perchè una nuova tipologia d'uva può essere appena stata acquistata e quindi non è stato prodotto ancora nessun vino.}
\end{verse}


\begin{itemize}
	\item \underline{MateriaPrima - Fornitore}: \textbf{Fornitura}*
	
	\begin{itemize}
		\item La Materia prima viene fornita da un solo tipo di fornitore (1,1)
		\item Un fornitore  fornisce da una o più tipologie di materie prime (1,N)
	\end{itemize}
	
\end{itemize}

\begin{verse}
*\emph{Nella relazione sono presenti gli attributi \textbf{DataAcquisto, Prezzo, Quantità} perchè una fornitura può essere richiesta più volte in data, prezzo e quantità diverse}
\end{verse}


\begin{itemize}
	\item \underline{BottigliaDiVino - Vino}: \textbf{TipoVino}
	
	\begin{itemize}
		\item La bottiglia di vino contiene una sola tipologia di vino (1,1)
		\item Un vino può essere contenuto in nessuna o più bottiglie di vino (0,N)*
	\end{itemize}
	
\end{itemize}

\begin{verse}
	*\emph{(0,N) perchè una tipologia di vino può essere appena stato prodotto e quindi non è stato ancora imbottigliato.}
\end{verse}

\begin{itemize}
	\item \underline{BottigliaDiVino - Tappo}: \textbf{TipoTappo}
	
	\begin{itemize}
		\item La bottiglia di vino ha una sola tipologia di tappo (1,1)
		\item Un tipo di tappo può appartenere a nessuna o più bottiglie di vino (0,N)*
	\end{itemize}
	
\end{itemize}

\begin{verse}
	*\emph{(0,N) perchè una tipologia di tappo può essere appena stato acquistato e quindi non è stato ancora assegnato a nessuna bottiglia di vino.}
\end{verse}

\begin{itemize}
	\item \underline{BottigliaDiVino - Bottiglia}: \textbf{TipoBottiglia}
	
	\begin{itemize}
		\item La bottiglia di vino è formata da una sola tipologia di bottiglia (1,1)
		\item Un tipo di bottiglia può appartenere a nessuna o più bottiglie di vino (0,N)*
	\end{itemize}
	
\end{itemize}

\begin{verse}
	*\emph{(0,N) perchè una tipologia di bottiglia può essere appena stata acquistata e quindi non è stata ancora asseganta a nessuna bottiglia di vino.}
\end{verse}

\begin{itemize}
	\item \underline{BottigliaDiVino - Magazzino}: \textbf{Conservata}
	
	\begin{itemize}
		\item Una tipologia di bottiglia di vino è conservata in un unico magazzino (1,1)
		\item Un magazzino contiene solo una tipologia di vino (1,1)
	\end{itemize}
	
\end{itemize}

\begin{itemize}
	\item \underline{BottigliaDiVino - Ordine}: \textbf{Contenuto}*
	
	\begin{itemize}
		\item Un tipo di bottiglia di vino può appartenere a nessuno o pù ordini di vendita (0,N)
		\item Un ordine di vendita contiene da uno a più tipi di bottiglie di vino (1,N)
	\end{itemize}
	
\end{itemize}

\begin{verse}
	*\emph{Nella relazione è presente l'attributo \textbf{QuantitàBottiglie} perchè serve sapere il numero di bottiglie di vino per tipologia vendute nell'ordine}
\end{verse}

\begin{itemize}
	\item \underline{Ordine - Corriere}: \textbf{Spedizione}*
	
	\begin{itemize}
		\item Un ordine viene spedito da un singolo corriere (1,1)
		\item Un corriere può spedire da uno a molteplici ordini (1,N)
	\end{itemize}
	
\end{itemize}

\begin{verse}
	*\emph{Nella relazione sono presenti gli attributi \textbf{DataSpedizione, Prezzo, DataArrivo} perchè dell'ordine è importante tracciare la data in cui è stato spedito l'ordine, il prezzo della spedizione e la data di consegna dell'ordine}
\end{verse}

\begin{itemize}
	\item \underline{Ordine - Acquirente}: \textbf{Venduta}
	
	\begin{itemize}
		\item Un ordine viene venduto ad un singolo acquirente (1,1)
		\item Un acquirente può effettuare da uno a più ordini (1,N)
	\end{itemize}
	
\end{itemize}

\begin{itemize}
	\item \underline{NegozioInterno - Partecipante - Evento}: \textbf{Ospita}*
	
	\begin{itemize}
		\item Un negozio interno può ospitare o nessuno o molteplici eventi (0,N)
		\item Un partecipante può partecipare ad un solo evento in un solo negozio(1,1)*
		\item Un evento può essere ospitato in uno o in molteplici negozi(1,N)
	\end{itemize}
	
\end{itemize}

\begin{verse}
	*\emph{Nella relazione è presente l'attributo \textbf{Data} perchè un evento può essere ospitato in un negozio, in una sola data. Per questo motivo il partecipante assiste ad un solo evento in una determinata data}
\end{verse}

\begin{itemize}
	\item \underline{Evento - Vino}: \textbf{TemaVino}
	
	\begin{itemize}
		\item Un evento può avere da uno a molteplici tipo di vino come tema (1,N)
		\item Un vino può essere essere il tema di nessuno o molteplici eventi(0,N)
	\end{itemize}
	
\end{itemize}

\begin{itemize}
	\item \underline{ProduzioneVino - Vino}: \textbf{Nasce}
	
	\begin{itemize}
		\item Delle determinate fasi fanno nascere un tipo particolare di vino (1,1)
		\item Un vino può nascere da una sola e specifica catena di produzione(1,1)
	\end{itemize}
	
\end{itemize}

\begin{itemize}
	\item \underline{BottigliaDiVino - Magazzino}: \textbf{Conservata}
	
	\begin{itemize}
		\item Una tipologia di bottiglia di vino è conservata in un unico magazzino (1,1)
		\item Un magazzino contiene solo una tipologia di vino (1,1)
	\end{itemize}
	
\end{itemize}


\begin{itemize}
	\item \underline{LineaProduttiva - Dipendente}: \textbf{Turno}*
	
	\begin{itemize}
		\item In una linea produttiva possono lavorare da uno a molteplici dipendenti (1,N)
		\item Un dipendete può lavorare in una sola linea produttiva (1,1)
	\end{itemize}
	
\end{itemize}

\begin{verse}
	*\emph{Nella relazione sono presenti gli attributi \textbf{InizioTurno, FineTurno} perchè è utile rappresentare il giorno e l'orario in cui un dipendente lavora. Entrambi gli attributi sono rappresentati attraverso il tipo DATETIME}
\end{verse}

\begin{itemize}
	\item \underline{LineaProduttiva - Dipendente}: \textbf{Diretto}
	
	\begin{itemize}
		\item Una linea produttiva è diretta da un solo dipendente (1,1)
		\item Un dipendete può dirigere o nessuna o più linee produttive (0,N)
	\end{itemize}
	
\end{itemize}

\begin{itemize}
	\item \underline{LineaProduttiva - Macchinario}: \textbf{Utilizzo}
	
	\begin{itemize}
		\item In una linea produttiva possono esserci uno a molteplici macchinari (1,N)
		\item Un macchinario può appartenere ad una singola linea produttiva (1,1)
	\end{itemize}
	
\end{itemize}

\begin{itemize}
	\item \underline{Macchinario - AziendaManutenzione}: \textbf{Manutenzione}*
	
	\begin{itemize}
		\item Un macchinario può ricevere o nessuna o molteplici manutenzioni (0,N)
		\item Un'azienda di manutenzione può revisionare uno o molteplici macchinari (1,N)
	\end{itemize}
	
\end{itemize}

\begin{verse}
	*\emph{Nella relazione è presente l'attributo \textbf{Costo} che identifica il costo della singola manutenzione su un determinato macchinario fatto da una determinata azienda}
\end{verse}

\begin{itemize}
	\item \underline{Dipendente - Dipendente}: \textbf{Referisce}
	
	\begin{itemize}
		\item Un dipendente può essere riferito da uno o molteplici altri dipendenti (1,N)
		\item Un dipendente può riferire o nessuno o un dipendente (0,1)
	\end{itemize}
	
\end{itemize}