L'interesse principale di questo progetto e' modellare una base di dati che supporti l'operativita' della cantina vinicola \emph{WineBase}. Il principale prodotto trattato e' la \textbf{bottiglia di vino} che e' la composizione una bottiglia di vetro, di un tappo e di un tipo di vino. Ogni bottiglia è identificata dal nome del vino e dall'anno (\emph{annata}) di imbottigliamento. Per ogni prodotto e' necessario tenere traccia del prezzo, del numero di unità prodotte e vendute, oltre che del tipo di tappo e del tipo di bottiglia.
Relativamente ai tipi di \textbf{vino}, identificati tramite il nome, rappresentiamo la gradazione alcolica, il tempo di fermentazione e l'uva utilizzata oltre che lo stato produttivo, ovvero se il vino viene prodotto ancora o è, ad oggi, fuori commercio. E' inoltre importante registrare la classificazione del vino: vini a denominazione d'origine controllata e garantita (D.O.C.G.); vini a denominazione d'origine controllata (D.O.C.); vini ad indicazione geografica tipica (I.G.T.).\\ Ovviamente la cantina acquista da altre aziende le materie prime che vanno a formare il prodotto finale. Le \textbf{materie prime} che vengono acquistate vengono identificate attraverso un ID univoco. Se la materia prima è l'uva, viene rappresentata dall'anno di raccolta, dal nome e dal colore. Se, invece, la materia prima sono i tappi, questa viene rappresentata dalla forma e dal materiale, infine se la materia prima sono le bottiglie, viene rappresentata dalla capacità e dal colore. Inoltre è presente il dato che identifica la quantità posseduta dalla cantina per i tappi e per le bottiglie, ma non per l'uva poiche' tutta l'uva viene utilizzata immediatamente dopo la ricezione.\\
Un tipo d'uva proviene da un \textbf{vigneto}, rappresentato da un ID univoco e dalla sua posizione geografica. \\
Le \textbf{aziende} con cui la cantina ha relazioni sono molteplici, e si dividono in \emph{fornitori}, \emph{aziende di manutenzione macchinari} e \emph{negozi interni} della cantina. Le aziende sono identificate univocamente attraverso la loro Partita IVA e sono caratterizzate dal nome commerciale, dal cognome e nome del referente oltre che dal numero di telefono, dall'email aziendale e dalla sua posizione geografica (che corrisponde con l'indirizzo di spedizione). Per le aziende che sono fornitori rappresentiamo poi la tipologia di materia prima fornita, sia essa uva, tappi o bottiglie.\\
Ovviamente le bottiglie di vino devono poter essere spedite in una determinata quantità in un \textbf{ordine} di vendita. Relativamente ad un ordine rappresentiamo un ID univoco, la data in cui è stato effettuato, il \textbf{corriere} (rappresentato tramite un ID ed un nome commerciale.) a cui è stata affidata la consegna, la data di spedizione, l'eventuale data di consegna e il prezzo totale (che deve corrispondere al prodotto tra la quantita' delle bottiglie di vino acquistate ed il relativo prezzo, sommato al costo della spedizione).\\
Ogni ordine viene effettuato da un acquirente, che può essere un'\textbf{azienda} o un \textbf{privato}. Per i privati, identificati tramite un ID, rappresentiamo il nome, il cognome, il numero di telefono e l'email oltre che l'indirizzo di residenza (che corrisponde con l'indirizzo di spedizione).\\
E' consuetudine, per la cantina \emph{WineBase}, organizzare degli eventi all'interno dei propri negozi. Gli \textbf{eventi} hanno una tematica che è rappresentata da un nostro vino e vengono definiti tramite il titolo dell'evento e il numero di edizione. Ogni edizione di un evento ha una data prefissata in cui si svolgerà o si e' svolto.
Riguardo al reparto produttivo sono presenti tutte le fasi di lavorazione dell'uva per la produzione del vino. Queste \textbf{linee produttive} sono l'ingresso delle materie prime, la pigiatura, la fermentazione, la vinificazione e la svinatura. Oltre a queste sono presenti altri reparti produttivi come l'imbottigliamento e il magazzino dove le bottiglie di vino vengono divise per la colorazione del vino in esse contenuto (rosso, bianco, rosato e spumante). Nel magazzino viene tenuta traccia della quantità delle bottiglie conservate. Tutti questi reparti sono identificati tramite un identificativo univoco.\\
In ognugno di questi reparti lavorano dei \textbf{dipendeti}. Per ogni dipendete (identificato tramite il codice fiscale) viene registrato il nome e il cognome. Ogni dipendente (esclusi alcuni supervisori) ha un supervisore a cui fare riferimento e ogni linea produttiva e' diretta da un supervisore. Per ogni dipendente e' necessario tenere traccia dei turni di lavoro effettuati e della linea produttiva in cui i vari turni sono svolti.\\
Per ogni linea produttiva sono presenti dei \textbf{macchinari} che aiutano i dipendenti nelle mansioni quotidiane. I macchinari sono identificati tramite un ID univoco. Ad ogni macchinario viene fatta una manutenzione periodica, è quindi utile tenere traccia della data della prossima manutenzione. I macchinari avranno anche un nome rappresentatativo e la data di acquisto.\\
Le manutenzioni hanno un costo e vengono eseguite in una determinata data. Le manutenzioni vengono eseguite da aziende, rappresentate come sopra.