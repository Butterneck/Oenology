Si vuole realizzare una base di dati per rappresentare al meglio il lavoro della cantina \epmh{Nome cantina}. Come prodotto principale c'è la bottiglia di vino che è caratterizzata dal tipo di vino contenuto, dall'annata, dal prezzo per bottiglia, dal numero di bottiglie prodotte e vendute ed infine dalla classificazione: vini a denominazione d'origine controllata e garantita (D.O.C.G.); vini a denominazione d'origine controllata (D.O.C.); vini ad indicazione geografica tipica (I.G.T.). Ciascuna tipologia di bottiglia di vino è immagazzinata in magazzini aventi il numero di bottiglie contenute divisi per la colorazione del vino (rosso, bianco, rosato e spumante).
Altro dato importante è utile identificare il vino prodotto, che viene rappresentato tramite il nome, la gradazione alcolica, il tempo di fermentazione, la tipologia di uva con cui è stata prodotta, caratterizzata dalla vedemmia (data raccolta) e dal colore. Importante inoltre tenere traccia se un determinato vino è ancora in fase di produzione o è diventato fuori commercio. 