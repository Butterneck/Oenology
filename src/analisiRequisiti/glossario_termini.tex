Considerazioni

\begin{center}
	\begin{longtable}{P{2cm}P{2cm}P{6cm}P{4cm}}
		\toprule
		\rowcolor[rgb]{.929, .929, .929} Termine & Sinonimi & Descrizione & Collegamento \\
		
		\midrule
		Bottiglia di vino & & Una bottiglia di vino della cantina in vendita online e nei negozi fisici. & Vino, Bottiglia, Tappo, Ordine, Magazzino\\
		\midrule
		Vino & Tipo Vino & Una tipologia di vino prodotta all'interno della cantina. & Uva, Evento, Bottiglia di vino \\
		
		\midrule
		Materia prima & & Una materia prima che viene acquistata e manipolata per comporre le bottiglie di vino. &  Vino, Tipo Uva, Fornitore, Bottiglia di vino\\
		
		\midrule
		Uva & Tipo Uva & Una tipologia di uva che viene acquistata per produrre il vino. &  Vigneto, Vino\\
		
		\midrule
		Vigneto & & Rappresenta il vigneto dalla quale proviene un determinato lotto di uva acquistato. &  Tipo Uva\\

		\midrule
		Acquirente & Privato, Azienda & Colui che ha effettuato l'ordine. E' un privato o un'azienda &  Ordine\\

		\midrule
		Azienda & Fornitore, Negozio interno, Az. Manutenz. & Un'azienda con cui la cantina ha rapporti, siano questi di fornitura, di vendita o di altra natura. &  Materia prima, Evento, Macchinario\\

		\midrule
		Ordine & & Un ordine effettuato da un acquirente, contiene le informazioni della merce acquistata e della spedizione. &  Bottiglia di vino, Corriere, Acquirente\\
		
		\midrule
		Corriere & & Un corriere che si occupa della spedizione di un ordine & Ordine\\
		\midrule
		
		Evento & & Un evento organizzato all'interno di un negozio interno in onore di uno o piu' vini prodotti dalla cantina. &  Tipo Vino, Negozio Interno\\

		\midrule
		Linea produttiva & & Un reparto produttivo interno alla cantina. &  Dipendente, Macchinario\\

		\midrule
		Dipendente & & Colui che lavora per la cantina ed è assegnato ad una linea produttiva &  Linea produttiva\\

		\midrule
		Macchinario & & Strumento meccanico o meccatronico utilizzato all'interno di una linea produttiva &  Linea produttiva, Azienda manutenzione\\
		
		\bottomrule
	\end{longtable}
\end{center}