Oltre agli identificatori visti nella sezione~\ref{analisi_entita}, sono stati aggiunti nuovi identificatori primari:

\begin{itemize}
	\item Si è deciso di aggiungere l'attributo \textbf{ID} (che funge da chiave primaria e referenziale con l'attributo \textbf{ID} di \emph{Azienda}) all' entità \emph{Fornitore};
	\item All'entità \emph{Privato} è stato aggiunto l'attributo \textbf{ID} che ha il ruolo di chiave primaria e di chiave referenziale con l'attributo \textbf{ID} di \emph{Acquirente};
	\item L'entità \emph{NegozioInterno} possiede un attributo \textbf{ID} che è chiave primaria e chiave referenziale con l'attributo \textbf{ID} di \emph{Azienda};
	\item L'entità \emph{Informazione} appena aggiunta ha come chiave primaria l'attributo \textbf{ID}
	\item L'entità \emph{Indirizzo} appena aggiunta ha come chiave primaria l'attributo \textbf{ID} che è anche chiave referenziale con l'attributo \textbf{ID} di \emph{Informazione o Vigneto};
	\item Le entità \emph{Uva, Tappo, Bottiglia} hanno ereditato l'attributo \textbf{ID} dalla generalizzazione \emph{MateriaPrima};
	\item E' stato aggiunto l'attributo \textbf{ID}, che è chiave primaria, a \emph{BottgiliaDiVino ed Evento} per rendere più semplice le relazioni con queste entità.
\end{itemize}