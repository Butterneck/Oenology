Oltre agli identificatori visti nella sezione~\ref{analisi_entita}, sono stati aggiunti nuovi identificatori primari:

\begin{itemize}
	\item Si è deciso di aggiungere l'attributo \textbf{ID} (che funge da chiave primaria) alle entità figlie \emph{FornitoreUva, FornitoreTappi, FornitoreBottiglie};
	\item Alle entità \emph{Azienda e Priavto} è stato aggiunto l'attributo \textbf{ID} che ha il ruolo di chiave primaria e di chiave referenziale con l'attributo \textbf{ID} di \emph{Acquirente};
	\item L'entità \emph{NegozioInterno} avrà un attributo \textbf{ID} che sarà chiave primaria e chiave referenziale con l'attributo \textbf{ID} di \emph{Azienda};
<<<<<<< HEAD
	\item L'entità \emph{Indirzzo} appena aggiunta avrà come chiave primaria l'attributo \textbf{ID} che sarà anche chiave referenziale o con l'attributo \textbf{ID} di \emph{Acquirente o Vigneto};
	\item E' stato aggiunto l'attributo \textbf{ID} a \emph{Uva, Tappo, Bottiglia, Evento} per rendere più semplice le relazioni con queste entità.
=======
	\item L'entità \emph{Indirzzo} appena aggiunta avrà come chiave primaria l'attributo \textbf{ID} che sarà anche chiave referenziale o con l'attributo \textbf{ID} di \emph{Acquirente o Vigneto}.
>>>>>>> 6863479d62ba918292c1791124e3416e81e12a39
\end{itemize}