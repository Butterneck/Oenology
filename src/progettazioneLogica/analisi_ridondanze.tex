Lo schema concettuale presenta 2 attributi ridondanti, cioè l'attributo \textbf{Annata} presente nell'entità \textbf{BottigliaDiVino} e l'attributo \textbf{PrezzoTotale} presente nell'entità \emph{Ordine}. Quest'ultimo, infatti, è derivabile da $\sum_{n = 1}^{j} (PrezzoBottiglia_n \times QuantitaBottiglie_n) + PrezzoSpedizione$, 
dove $j$ è il numero di tipologie di bottiglie di vino presenti nell'ordine,
\textbf{PrezzoBottiglia} è l'attributo dell'entità \emph{BottigliaDiVino}, \textbf{PrezzoSpedizione} è l'attributo della relazione \emph{Spedizione} e \textbf{QuantitaBottiglie} è l'attributo della relazione \emph{Dettaglio}. L'operazione coinvolta è la \emph{Stampa resoconto guadagno} (Esempio nella sezione \ref{stampa_resoconto_guadagno}), che avviene 1 volta alla settimana. Supponiamo quindi di avere una tavola dei volumi come nella tabella riportata qui sotto.
Analizziamo la tavola dei volumi e delle operazioni (caso migliore con volume \emph{Dettaglio} uguale a volume \emph{Ordine}):

\begin{center}
	\begin{tabular}{P{2cm}P{8cm}P{4cm}}
		\multicolumn{3}{c}{\textbf {\large {Tavola dei volumi}}} \\
		\toprule
		\rowcolor[rgb]{.929, .929, .929} Concetto & Costrutto & Volume \\
		\midrule
		BottigliaDiVino & E & 50\\
		\midrule
		Ordine & E & 350\\
		\midrule
		Corriere & E & 25\\
		\midrule
		Dettaglio & R & 350\\
		\midrule
		Spedizione & R & 350\\
		\bottomrule
	\end{tabular}

	\vspace{0.5cm}

	\textbf{\large{Tavola delle operazioni}}\\
	\vspace{0.2cm}
	\begin{minipage}{6cm}
		\rightline{
			\begin{tabular}{P{2cm}P{2cm}P{1.1cm}P{1cm}}
				\multicolumn{4}{c}{\textbf {Senza ridondanza}} \\
				\toprule
				\rowcolor[rgb]{.929, .929, .929} Concetto & Costrutto & Accesso & Tipo \\
				\midrule
				BottigliaDiVino & E & 350 & L\\
				\midrule
				Ordine & E & 1 & L\\
				\midrule
				Dettaglio & R & 350 & L\\
				\midrule
				Spedizione & R & 350 & L\\
				\bottomrule
			\end{tabular}
		}
	\end{minipage}
	\hspace{2mm}
	\begin{minipage}{6cm}
		\leftline{
			\begin{tabular}{P{2cm}P{2cm}P{1.1cm}P{1cm}}
				\multicolumn{4}{c}{\textbf {Con ridondanza}} \\
				\toprule
				\rowcolor[rgb]{.929, .929, .929} Concetto & Costrutto & Accesso & Tipo \\
				\midrule
				BottigliaDiVino & E & 0 & -\\
				\midrule
				Ordine & E & 1 & L\\
				\midrule
				Dettaglio & R & 0 & -\\
				\midrule
				Spedizione & R & 0 & -\\
				\bottomrule
			\end{tabular}
		}
	\end{minipage}
	
\end{center}

\begin{flushleft}
In questo esempio vediamo che nello schema \emph{con ridondanza} viene effettuato 1 accesso e 350 letture a settimana alla tabella \emph{Ordini} per ottenere il resoconto settimanale (semplice somma di tutti i prezzi totali degli ordini di una specifica settimana).\\
\vspace{0.5cm}
Nello schema \emph{senza ridondanza}, invece, per ogni ordine occorrerà accedere alle seguenti tabelle:
\begin{itemize}
	\item \emph{BottiglieDiVino} per conoscere il prezzo della singola bottiglia dell'ordine;
	\item \emph{Spedizioni} per conoscere il prezzo della spedizione;
	\item \emph{Dettagli} per conoscere la quantità di bottiglie acquistate di un determinato tipo.
\end{itemize}
Si avranno $(350\times3)\times350$, dove $(350\times3)$ sono gli accessi totali alle tre tabelle citate prima e 350 sono le letture della tabella \emph{Ordini}. In totale si avranno $367500$ letture.\\
<<<<<<< HEAD
=======
\begin{verse}
	Alla luce dei risultati emersi della precedente analisi si è deciso di mantenere l'attributo \textbf{Prezzo totale} in \emph{Ordini}. Inoltre si è deciso di mantenere l'attributo \textbf{Annata} in \emph{BottigliaDiVino} perchè il ragionamento ed il carico è analogo alla suddetta situazione.
\end{verse} 
>>>>>>> 3c80bc4750c85057c30313b686631f981adf0aa6
\end{flushleft}
\emph{
	Alla luce dei risultati emersi della precedente analisi si e' deciso di mantenere l'attributo \textbf{Prezzo totale} in \emph{Ordini}. Inoltre si e' deciso di mantenere l'attributo \textbf{Annata} in \emph{BottigliaDiVino} perche' il ragionamento ed il carico e' analogo alla suddetta situazione. Il caso di studio appena presentato rappresenta inoltre il caso migliore in presenza di 350 ordini, ovvero il caso in cui ogni ordine contiene solamente un tipo di bottiglia di vino. Normalmente il volume del concetto \emph{Dettaglio} sarebbe di gran lunga superiore, rendendo, di fatto, la scelta presa ancor piu' giustificata.
}
