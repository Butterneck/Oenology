Lo schema concenttuale presenta numerose generalizzazioni. Si procede quindi all'analisi di queste ultime per permettere la traduzione verso lo schema logico.
\begin{flushleft}
	\textbf{\large{MateriaPrima}}\\
	L'entità \emph{MateriaPrima} è una generalizzazione completa delle entità \emph{Uva, Bottiglia, Tappo}. Queste ultime presentano un carico di attributi e di relazioni maggiore rispetto a \emph{MateriaPrima}. Si è deciso di eliminare la generalizzazione e di creare una relazione specifica per ogniuno tra \emph{Uva, Bottiglia, Tappo} con l'entità \emph{Fornitore}.
\end{flushleft}

\begin{flushleft}
	\textbf{\large{Fornitore}}\\
	L'entità \emph{Fornitore} non ha attributi ed è generalizzata dall'entità \emph{Azienda}. A sua volta \emph{Fornitore} è una generalizzazione di \emph{FornitoreUva, FornitoreTappi, FornitoreBottiglie}. Si è deciso di mantenere l'entità \emph{Fornitore} (avendo relazioni specifiche) e di eliminare \emph{FornitoreUva, FornitoreTappi, FornitoreBottiglie}. E' stato quindi aggiunto alla sopracitata entita' l'attributo \textbf{Tipologia} per identificare il tipo di fornitura di ogni fornitore.
\end{flushleft}


\begin{flushleft}
	\textbf{\large{Acquirente}}\\
	L'entità \emph{Acquirente} presenta un attributo ed una relazione. A loro volta anche le entità figlie \emph{Azienda e Privato} hanno attributi significativi. Per ristrutturare questa generalizzazione si è deciso di creare due relazioni differenti. La prima si chiama \textbf{IsAzienda} dove la cardinalità da \emph{Azienda} ad \emph{Acquirente} è $(0,1)$ poichè un'azienda può non essere un acquirente, mentre da \emph{Acquirente} ad \emph{Azienda} è $(0,1)$ poichè non tutti gli acquirenti sono un'azienda.
	Con una logica analoga a quella applicata precedentemente e' stata creata la relazione \textbf{IsPrivato}, ovvero la cardinalità da \emph{Acquirente} a \emph{Privato} è $(0,1)$ poichè non tutti gli acquirenti sono privati, mentre da \emph{Privato} ad \emph{Acquirente} è $(1,1)$ poichè un privato è sempre un acquirente. Le entità \emph{Azienda} e \emph{Privato} mantengono gli attributi visti in precedenza, in più ad \emph{Azienda} verrà aggiunto l'attributo \emph{NumAcquirente} che sarà chiave esterna ad \emph{Acquirente}. Si è deciso di aggiungere questo attributo con la consapevolezza che quest'ultimo corrispondera' a \textbf{NULL} in corrispondenza dei record che rappresentano fornitori che non sono anche acquirenti, ma poiche' la quantita' di aziende acquirenti ci si aspetta sia di gran lunga superiore alla quantita' dei fornitori non acquirenti, ci si aspetta che i campi con valore \emph{NULL} siano in numero irrilevante.
\end{flushleft}

\begin{flushleft}
	\textbf{\large{Azienda}}\\
	L'entità \emph{Azienda} può identificare un fornitore, un'azienda di manutenzione o un negozio interno. Per ristrutturare questa generalizzazione si è deciso di aggiungere una relazione chiamata \emph{IsNegozioInterno} con cardinalità $(0,1)$ da \emph{Azienda} a \emph{NegozioInterno} poichè non tutte le aziende sono un \emph{NegozioInterno}. La cardinalità da \emph{NegozioInterno} ad \emph{Azienda} è $(1,1)$ poichè un negozio interno è anch'esso un'azienda. è stata scelta questa soluzione perchè il numero di aziende che non sono negozi interni è maggiore rispetto al numero dei negozi interni, perciò è stata esclusa l'aggiunta di un attributo in \emph{Azienda} che identifichi se quest'ultima è anche negozio interno. La stessa soluzione è stata adottata anche per la generalizzazione con \emph{Fornitore}. Nella generalizzazione di \emph{AzManutenzione}, invece, si è deciso di togliere \emph{AzManutenzione} e di non specificare se un'azienda è un' azienda che effettua la manutenzione dei macchinari, ma di ricavarlo tramite la relazione \emph{Manutenzione}.
\end{flushleft}

\begin{flushleft}
	\textbf{\large{LineaProduttiva}}\\
	L'entità \emph{LineaProduttiva} è una generalizzazione a più livelli, infatti essa generalizza le entità \emph{ProduzioneVino, Imbottigliamento, Magazzino}. A sua volta l'entità \textbf{ProduzioneVino} generalizza \emph{IngrMateriePrime, Pigiatura, Fermentazione, Vinificazione, Svinatura}. Infine l'entità \emph{Magazzino} generalizza le entità \emph{MagBianco, MagSpumante, MagRosato, MagRosso}. Essendo tutte generalizzazioni complete e non avendo attributi (esclusa l'entità \emph{Magazzino}*) che vanno a particolarizzare le varie entità figlie, si è deciso di aggiungere un attributo \textbf{Nome} nell'entità \emph{LineaProduttiva} che identifica queste specializzazioni.
\end{flushleft}

\begin{verse}
	*L'attributo \textbf{NumBottiglie} dell'entità \emph{Magazzino}, essendo di cardinalità $(1,1)$ per l'entità \emph{BottigliaDiVino}, è stato assegnato a quest'ultima rinominandolo \textbf{NumBottiglieMagazzino} perchè viene identificato univocamento per ogni tipologia di bottiglia di vino.
\end{verse}