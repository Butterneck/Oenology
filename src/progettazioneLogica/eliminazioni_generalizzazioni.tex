Lo schema concenttuale presenta molte generalizzazioni. Si procede all'analisi di quest'ultime per permettere la traduzione verso lo schema logico.\\

\begin{flushleft}
	\textbf{\large{Fornitore}}\\
	L'entità \emph{Fornitore} presenta solo due attributi comuni a tutte le entità figlie, cioè l'identificatore \textbf{ID} e il \textbf{Nome}. Inoltre l'entità padre \emph{Fornitore} non presenta nessuna relazione con nessun' altra e entità, cose che invece avviene per tutte e tre le entità figlie. Essendo una generalizzazione completa si è deciso di eliminare l'entità padre \emph{Fornitore}. 
\end{flushleft}


\begin{flushleft}
	\textbf{\large{Acquirente}}\\
	L'entità \emph{Acquirente} presenta molteplici attributi ed una relazione. A loro volta anche le entità figlie \emph{Azienda e Privato} hanno attributi significativi. Per ristrutturare questa generalizzazione si è deciso di creare due relazioni differenti. La prima si chiama \textbf{IsAzienda} dove la cardinalità da \emph{Azienda ad Acquirente} è (1,1) perchè un'azienda è sempre un'acquirente, invece da \emph{Acquirente ad Azienda} è (0,1) perchè non sempre un acquirente è un'azienda.
	La seconda relazione si chiama \textbf{IsPrivato} dove le cardinalità sono rappresentate come fra \emph{Azienda ed Acquirente}. Le entità \emph{Azienda e Privato} avranno sempre gli attributi visti in precedenza.
\end{flushleft}

\begin{flushleft}
	\textbf{\large{Azienda}}\\
	L'entità \emph{Azienda} può essere un negozio interno della cantina. Per ristrutturare questa generalizzazione si è deciso di aggiungere una relazione chiamata \emph{IsNegozioInterno} con cardinalità (0,1) da \emph{Azienda a NegozioInterno} perchè un'azienda può o non essere un \emph{NegozioInterno}. La cardinalità da \emph{NegozioInterno a Azienda} è (1,1) perchè un negozio interno è anch'egli un'azienda. Si è scelta questa strada perchè il carico di aziende che non sono negozi interni è maggiore rispetto a quelle che sono negozi interni, perciò è stato escluso l'aggiunta di un attributo in \emph{Azienda} che identificava se quest'ultima era anche negozio interno.
\end{flushleft}

\begin{flushleft}
	\textbf{\large{LineaProduttiva}}\\
	L'entità \emph{LineaProduttiva} è una generalizzazione a più livelli, infatti essa generalizza le entità \emph{ProduzioneVino, Imbottigliamento, Magazzino}. A sua volta l'entità \textbf{ProduzioneVino} generalizza \emph{IngrMateriePrime, Pigiatura, Fermentazione, Vinificazione, Svinatura}. Infine l'entità \emph{Magazzino} generalizza le entità \emph{MagBianco, MagSpumante, MagRosato, MagRosso}. Essendo tutte generalizzazioni complete e non avendo attributi (a parte l'entità \emph{Magazzino}*) che vanno a particolarizzare le varie entità figlie, si è deciso di aggiungere un attributo \textbf{Tipologia} nell'entità \emph{LineaProduttiva} che identifica queste specializzazioni.
\end{flushleft}

\begin{verse}
	*L'attributo \textbf{NumBottiglie} dell'entità \emph{Magazzino}, essendo di cardinalità (1,1) per l'entità \emph{BottigliaDiVino}, è stato assegnato a quest'ultima rinominandolo \textbf{NumBottiglieMagazzino} perchè viene identificato univocamento per ogni tipologia di bottiglia di vino.
\end{verse}