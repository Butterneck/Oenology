Lo schema concenttuale presenta molte generalizzazioni. Si procede all'analisi di queste ultime per permettere la traduzione verso lo schema logico.\\


\begin{flushleft}
	\textbf{\large{MateriaPrima}}\\
	L'entità \emph{MateriaPrima} è una generalizzazione completa delle entità \emph{Uva, Bottiglia, Tappo}. Queste ultime presentano un carico di attributi e di relazioni maggiore rispetto a \emph{MateriaPrima}. Si è deciso di eliminare la generalizzazione e di creare una relazione specifica per \emph{Uva, Bottiglia, Tappo} con l'entità \emph{Fornitore}.
\end{flushleft}

\begin{flushleft}
	\textbf{\large{Fornitore}}\\
	L'entità \emph{Fornitore} non ha attributi ed è generalizzata dall'entità \emph{Azienda}. A sua volta \emph{Fornitore} è una generalizzazione di \emph{FornitoreUva, FornitoreTappi, FornitoreBottiglie}. Si è deciso di mantenere l'entità \emph{Fornitore} (avendo relazioni specifiche) e di eliminare \emph{FornitoreUva, FornitoreTappi, FornitoreBottiglie}. E' quindi stato aggiunto l'attributo \textbf{Tipologia} per identificare il tipo di fornitura di ogni fornitore.
\end{flushleft}


\begin{flushleft}
	\textbf{\large{Acquirente}}\\
	L'entità \emph{Acquirente} presenta un attributo ed una relazione. A loro volta anche le entità figlie \emph{Azienda e Privato} hanno attributi significativi. Per ristrutturare questa generalizzazione si è deciso di creare due relazioni differenti. La prima si chiama \textbf{IsAzienda} dove la cardinalità da \emph{Azienda} ad \emph{Acquirente} è $(0,1)$ poiche' un'azienda puo' non essere un acquirente, mentre da \emph{Acquirente} ad \emph{Azienda} è $(0,1)$ poiche' non tutti gli acquirenti sono un'azienda.
	La seconda relazione si chiama \textbf{IsPrivato} ed e' stata creata utilizzando lo stesso ragionamento di IsAzienda, ovvero la cardinalita' da \emph{Acquirente} a \emph{Privato} e' $(0,1)$ poiche' non tutti gli acquirenti sono privati, mentre da \emph{Privato} ad \emph{Acquirente} è $(1,1)$ poiche' un privato e' sempre un acquirente. Le entità \emph{Azienda} e \emph{Privato} mantengono gli attributi visti in precedenza, in piu' ad \emph{Acquirente} verra' aggiunto l'attributo \emph{NumAcquirente} che sara' chiave referenziale ad \emph{Acquirente}. Si e' deciso di aggiungere questo attributo consapevoli dell'eventualita' che quest'ultimo possa essere \textbf{NULL} visto che il numero di aziende che non sono acquirenti (es. fornitori) e' minore di quelle che lo sono.
\end{flushleft}

\begin{flushleft}
	\textbf{\large{Azienda}}\\
	L'entità \emph{Azienda} puo' identificare un fornitore, un'azienda di manutenzione o un negozio interno. Per ristrutturare questa generalizzazione si è deciso di aggiungere una relazione chiamata \emph{IsNegozioInterno} con cardinalità $(0,1)$ da \emph{Azienda} a \emph{NegozioInterno} poiche' non tutte le aziende sono un \emph{NegozioInterno}. La cardinalità da \emph{NegozioInterno} ad \emph{Azienda} è $(1,1)$ poiche' un negozio interno è anch'esso un'azienda. E' stata scelta questa soluzione perchè il numero di aziende che non sono negozi interni è maggiore rispetto al numero dei negozi interni, perciò è stata esclusa l'aggiunta di un attributo in \emph{Azienda} che identifichi se quest'ultima e' anche negozio interno. La stessa soluzione e' stata adottata anche per la generalizzazione con \emph{Fornitore}. Nella generalizzazione di \emph{AzManutenzione}, invece, si è deciso di togliere \emph{AzManutenzione} e di non specificare se un'azienda è un' azienda che effettua la manutenzione dei macchinari, ma di ricavarlo tramite la relazione \emph{Manutenzione}.
\end{flushleft}

\begin{flushleft}
	\textbf{\large{LineaProduttiva}}\\
	L'entità \emph{LineaProduttiva} è una generalizzazione a più livelli, infatti essa generalizza le entità \emph{ProduzioneVino, Imbottigliamento, Magazzino}. A sua volta l'entità \textbf{ProduzioneVino} generalizza \emph{IngrMateriePrime, Pigiatura, Fermentazione, Vinificazione, Svinatura}. Infine l'entità \emph{Magazzino} generalizza le entità \emph{MagBianco, MagSpumante, MagRosato, MagRosso}. Essendo tutte generalizzazioni complete e non avendo attributi (a parte l'entità \emph{Magazzino}*) che vanno a particolarizzare le varie entità figlie, si è deciso di aggiungere un attributo \textbf{Nome} nell'entità \emph{LineaProduttiva} che identifica queste specializzazioni.
\end{flushleft}

\begin{verse}
	*L'attributo \textbf{NumBottiglie} dell'entità \emph{Magazzino}, essendo di cardinalità $(1,1)$ per l'entità \emph{BottigliaDiVino}, è stato assegnato a quest'ultima rinominandolo \textbf{NumBottiglieMagazzino} perchè viene identificato univocamento per ogni tipologia di bottiglia di vino.
\end{verse}